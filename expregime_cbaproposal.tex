\documentclass[12pt]{article}
\usepackage[T1]{fontenc}
\usepackage{calc}
\usepackage{setspace}
\usepackage{multicol}
\usepackage{fancyheadings}

\usepackage{graphicx}
\usepackage{color}
\usepackage{rotating}
\usepackage{harvard}
\usepackage{aer}
\usepackage{aertt}
\usepackage{verbatim}

\setlength{\voffset}{-0.25in}
\setlength{\topmargin}{0pt}
\setlength{\hoffset}{-0.25in}

\setlength{\oddsidemargin}{0pt}
\setlength{\headheight}{0pt}
\setlength{\headsep}{0in}
\setlength{\marginparsep}{0pt}
\setlength{\marginparwidth}{0pt}
\setlength{\marginparpush}{0pt}

\setlength{\footskip}{.3in}
\setlength{\textwidth}{7in}
\setlength{\textheight}{9.5in}
\setlength{\parskip}{1pc}
\setlength{\parindent}{0pc}

\renewcommand{\baselinestretch}{1}

\newcommand{\bi}{\begin{itemize}}
\newcommand{\ei}{\end{itemize}}
\newcommand{\be}{\begin{enumerate}\setlength{\leftmargin}{0pt}}
\newcommand{\ee}{\end{enumerate}}
\newcommand{\bd}{\begin{description}}
\newcommand{\ed}{\end{description}}
\newcommand{\prbf}[1]{\textbf{#1}}
\newcommand{\prit}[1]{\textit{#1}}
\newcommand{\beq}{\begin{equation}}
\newcommand{\eeq}{\end{equation}}
\newcommand{\beqa}{\begin{eqnarray}}
\newcommand{\eeqa}{\end{eqnarray}}
\newcommand{\bdm}{\begin{displaymath}}
\newcommand{\edm}{\end{displaymath}}
\newcommand{\script}[1]{\begin{cal}#1\end{cal}}
\newcommand{\citee}[1]{\citename{#1} (\citeyear{#1})}
\newcommand{\h}[1]{\hat{#1}}
\newcommand{\ds}{\displaystyle}

\newcommand{\app}
{
\appendix
}

\newcommand{\appsection}[1]
{
\let\oldthesection\thesection
\renewcommand{\thesection}{Appendix \oldthesection}
\section{#1}\let\thesection\oldthesection
\renewcommand{\theequation}{\thesection\arabic{equation}}
\setcounter{equation}{0}
}

%\pagestyle{fancyplain}
%\lhead{}
%\chead{Identifying Regime Switching in Adaptive Expectations, Its Causes, and Consequences}
%\rhead{\thepage}
%\lfoot{}
%\cfoot{}
%\rfoot{}

\pagestyle{plain}


\begin{document}
\thispagestyle{empty}
\begin{center}
\textbf{CBA FACULTY/ACADEMIC STAFF PROPOSAL FUND APPLICATION}\\
\textbf{COVER SHEET}\\
\textbf{March 25, 2013}\\
\end{center}

PROPOSAL TITLE: \begin{center}Identifying Regime Switching in Adaptive Expectations, Its Causes, and Consequences\end{center}

\be
\item Primary Applicant:
  \be
  \item Name: James Murray
  \item Department: Economics
  \item Campus Address: 413 Wimberly Hall
  \item Campus Telephone: (608) 785-5140
  \item Job title: Assistant Professor
  \item Years of service at UW-L: 4 years.
  \ee
  Additional Applicants: None.
\item Total amount requested of the committee: \$5,000.
\item Has this or will this proposal be submitted to another potential funding source? No.
\ee
  
\newpage
\noindent \textbf{Abstract:} It is widely understood in the macroeconomics literature that market participants' expectations matter for future economic outcomes.  A literature that focuses on least-squares learning (an expectations mechanism where market participants use relatively simple statistical regression models to form forecasts) has found that adaptive expectations can explain important macroeconomic episodes, such as escalations of inflation and economic volatility, and persistence in contractions of employment and aggregate investment.  There has been relatively little work, though, investigating the specific structure that expectations mechanisms should take.  This paper will investigate the possibility that market participants may switch between alternative forecasting models, which may lead to sudden swings in expectations, and in turn have important consequences on the macroeconomy.  I will estimate a Markov-switching model on Survey of Professional Forecasters' expectations to identify regime switches, and use these estimates to determine possible economic factors leading to switching, and estimate the subsequent consequences on macroeconomic variables such as unemployment, real GDP, and inflation.

\noindent \textit{Keywords:} Learning, expectations, professional forecasts, regime switching. \\
\noindent \textit{JEL classification:} C13, E20, E31. 

\noindent \textbf{Objectives:}
\be
\vspace*{-1pc} \item The purpose of this research is to expand macroeconomists' knowledge of how market participants form expectations and what are the implications on the macroeconomy.  
\item Enhance my teaching of undergraduate macroeconomics by improving my understanding for the behavior of adaptive expectations in the U.S. economy and the implications for economic stability.
\ee

\vspace*{-1pc} \noindent \textbf{Outcome:} Quality research paper which will be distributed first through conference presentations and working paper series by Spring 2014 and eventually published in a peer-reviewed journal. 

\noindent \textbf{Impact:}
\be
\vspace*{-1pc} \item Macroeconomic literature:  The proposed research will contribute to a literature examining the role adaptive expectations plays on macroeconomic dynamics.   \citee{pigou} first suggested that expectations can cause business cycle fluctuations, and more recently adaptive expectations has been used to explain the run-up of inflation and volatility in the 1970s (\citee{ow2005b}, \citee{primiceri2006}), persistent contractions and expansions of employment and investment in the business cycle (\citee{eusepi_preston_aer2011}), and overall expectation dynamics have been shown to be a primary driver of business cycles (\citee{eudey_perli}, \citee{milani2011}).  \citee{branch_evans_2006} were the first to explore the specific structure of the adaptive expectations mechanism, looking carefully at the statistical models that are popular in the literature.  They find that the most popular modeling strategy for expectations (constant-gain, least-squares regressions) describes well actual professional forecasts reported by the Survey of Professional Forecasters.  \citee{markiewicz2013} extend this analysis to bond market and exchange rate variables and conclude the same.  Missing in this literature is an examination into whether there is sudden switching in expectations behavior, which can be used to explain sudden economic contractions or expansions, and changes in economic volatility.  \citee{eudey_perli} do find evidence for regime switching in professional forecasts (from pessimism to optimism and vice-versa) that is consistent with regime switching in economic growth, but they do not attempt to model how these expectations are formed.  The proposed research will fill this gap in the literature by estimating regime-switching behavior in terms of the statistical models that are used to form forecasts.  Furthermore, I will attempt to estimate the causes for these regime changes, and estimate the impact that regime changes in expectations has on the macroeconomy.

\item Personal research goals: This research is part of my broader research program to investigate the presence of adaptive expectations and uncertainty in the macroeconomy, and estimate its impact on the macroeconomy.  The proposed research complements one of my recently published papers on adaptive expectations and uncertainty regarding the conduct of monetary policy (\citee{herro_murray}).  Closely related to this is a paper I am currently preparing for submission that examines the impact of expectations regarding the conduct of fiscal policy (\citee{murray_fiscal}).  The proposed research also complements my papers that are currently under review, including \citee{murray_initexp} and \citee{murray_regime}, each which estimate the impact that adaptive expectations has on the U.S. economy in the context of popularly estimated dynamic, stochastic, general equilibrium (DSGE) models.

\item Target journals: \textit{Journal of Macroeconomics}, \textit{B.E. Journal of Macroeconomics}, \textit{Quantitative Economics}, \textit{Macroeconomic Dynamics}
\ee

\noindent \textbf{Description:}

The purpose of this research is to expand macroeconomists' understanding of how adaptive expectations are formed in the macroeconomy, and what the implications are for macroeconomic outcomes.  Specifically, I will look for evidence of regime-switching behavior in expectations formation in the U.S. economy, where market participants suddenly change the manner in which they form expectations.  A switch would imply a change in popular perception of how the U.S. economy evolves.  It is plausible that recent unprecedented events could have caused such regime changes.  Examples may include the September 11, 2001, terrorist attack, the 2007-2009 financial crises, and the prolonged and severe period on unemployment since the financial crisis.  It is possible, too, that unprecedented government economic policy could cause such switches in expectations.  Recent examples include the prolonged period of easy monetary policy (the Federal Reserve has kept its key interest rate near 0\% since late 2008) and the recent political crises over the federal budget that led to Standard and Poor's credit downgrading of U.S. government debt.  Less recently, the change in monetary policy beginning in 1979 could plausibly have caused regime changes in expectations.  It was then that Paul Volcker was appointed chairman of the Federal Reserve and quickly and significantly raised interest rates to combat inflation, even at the expense of causing economic contraction in an already struggling economy.

I will estimate potential regime changes by computing rolling window regression forecasts for common macroeconomic variables such as  inflation, unemployment, growth in real GDP, and growth in real GDP components of consumption and investment.  I will examine multiple functional forms for these regression models including the following:
\be
\item Decreasing gain least-squares regression (ordinary least squares)
\item Constant gain least-squares regression (weighted least squares, where most recent observations are given the greatest weight)
\item Constant gain and decreasing gain univariate forecasting models (forecast depends on only its lags and not on other macroeconomic data)
\ee
I will match these forecasts to data on the Survey of Professional Forecasters using a Markov-chain regime switching procedure (\citee{kim1994}) that supposes the chosen forecast can switch exogenously between the potential models.  The procedure assumes switching is exogenous, so it uses the information in the data on the survey forecasts and regression forecasts to determine the best fitting model in each time period.  Switching from a decreasing gain regression to one of the other two models will result in a sudden and likely significant change in expectations that likely has important implications on subsequent macroeconomic activity.  One contribution of the paper will be to identify when such changes have occurred, and describe the nature of these changes (i.e. report on which variable's forecast did the regime switching occur, to what regression model was the change, how large a shift in expectations occurred, etc).  The second contribution of the paper will be to align regime switching information with recent economic history of the United States, to determine whether significant economic events or changes in economic policy corresponded to changes in market participants' behavior in forming expectations.  The final contribution of the paper will be to use the estimates on regime changes in a vector autoregression (VAR) model with autoregressive conditional heterskedastic (ARCH) shocks (similar methodology as one of my previous papers, \citee{herro_murray}) to determine whether regime changes in expectations cause downturns or upturns in macroeconomic activity, and/or whether they cause changes in economic volatility.

I will produce a high quality paper to circulate at conferences and working paper series, and eventually publish in a peer reviewed journal.  The activities necessary to accomplish these goals, and the expected time frames, are the following:
\be
\item Write computer program in C (C is a computer programming language) to compute implied forecasts using the statistical models described above.\\
\textit{Time frame:} Complete by early Summer 2013.
\item Adapt existing computer programs that I have written in C (for previous work, \citee{murray_regime} and \citee{haupert_murray}) to estimate a Markov-chain regime switching model for expectations.\\
\textit{Time frame:} Complete by mid/late Summer 2013.
\item Write computer programs to estimate a VAR/ARCH model that includes estimated regime changes as explanatory variables\\
\textit{Time frame:} Complete late Summer 2013.
\item Write introduction, literature review, and motivation for the paper.\\
\textit{Time frame:} Complete by early Fall 2013.
\item Write methodology, results, and conclusion sections for the paper.\\
\textit{Time frame:} Complete by mid Fall 2013.
\item Circulate paper in working paper series sponsored by SSRN (Social Science Research Network) and at either the Southern Economic Association (SEA) 2013 annual conference (November, 2013) or the Midwest Economics Association (MEA) 2014 annual conference (March 2014).  \\
\textit{Time frame:} Complete by March 2014.
\item Consider feedback from the paper and submit it for publication by the end of Spring 2014 semester.
\ee

\noindent \textbf{Statement of Progress and/or Outcomes on Previously Funded Proposals:}\\
CBA Research Grant, ``Fiscal and Expenditure Multipliers When There Are Adaptive Expectations''
\noindent I narrowed the focus of this paper to fiscal multipliers, versus both fiscal and expenditure multipliers, and examined the consequences of expected versus unexpected fiscal policy under adaptive expectations.  I also used an alternative methodology that was more tractable, and that allowed me to focus on the implications of fiscal policy without making strong assumptions on the remaining structure of the macroeconomy.  I presented the paper at the 2012 Southern Economic Association Annual Meeting, and I am currently preparing a draft to submit for publication to Fiscal Studies.  

\noindent \textbf{Budget:}  Stipend = \$5,000.  Total = \$5,000.

\vspace*{-0.4in}
\begin{singlespace}
%\nocite{*}
\bibliographystyle{econometrica}
\bibliography{expregime.bib}
\end{singlespace}

\end{document}



